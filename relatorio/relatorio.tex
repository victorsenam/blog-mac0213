%%%%%%%%%%%%%%%%%%%%%%%%%%%%%%%%%%%%%%%%%
% Thin Sectioned Essay
% LaTeX Template
% Version 1.0 (3/8/13)
%
% This template has been downloaded from:
% http://www.LaTeXTemplates.com
%
% Original Author:
% Nicolas Diaz (nsdiaz@uc.cl) with extensive modifications by:
% Vel (vel@latextemplates.com)
%
% License:
% CC BY-NC-SA 3.0 (http://creativecommons.org/licenses/by-nc-sa/3.0/)
%
%%%%%%%%%%%%%%%%%%%%%%%%%%%%%%%%%%%%%%%%%

%----------------------------------------------------------------------------------------
%	PACKAGES AND OTHER DOCUMENT CONFIGURATIONS
%----------------------------------------------------------------------------------------

\documentclass[a4paper, 11pt]{article} % Font size (can be 10pt, 11pt or 12pt) and paper size (remove a4paper for US letter paper)
\usepackage[utf8]{inputenc}
\usepackage[protrusion=true,expansion=true]{microtype} % Better typography
\usepackage{graphicx} % Required for including pictures
\usepackage[brazilian]{babel}

\usepackage{mathpazo} % Use the Palatino font
\usepackage[T1]{fontenc} % Required for accented characters
\linespread{1.05} % Change line spacing here, Palatino benefits from a slight increase by default

\makeatletter
\renewcommand{\@listI}{\itemsep=0pt} % Reduce the space between items in the itemize and enumerate environments and the bibliography

\renewcommand{\maketitle}{ % Customize the title - do not edit title and author name here, see the TITLE block below
\begin{flushright} % Right align
{\LARGE\@title} % Increase the font size of the title

\vspace{50pt} % Some vertical space between the title and author name

{\large\@author} % Author name
\\\@date % Date

\vspace{40pt} % Some vertical space between the author block and abstract
\end{flushright}
}

%----------------------------------------------------------------------------------------
%	TITLE
%----------------------------------------------------------------------------------------

\title{\textbf{MAC0213 - RE}\\ % Title
Relatório Final} % Subtitle

\author{\textsc{Victor Sena Molero} % Author
\\{\textit{8941317}}} % Institution

\date{\today} % Date

%----------------------------------------------------------------------------------------

\begin{document}

\maketitle % Print the title section

%----------------------------------------------------------------------------------------
%	ESSAY BODY
%----------------------------------------------------------------------------------------

\section*{Pátio Digital}
O Pátio Digital é uma iniciativa da Secretaria Municipal de Educação (SME) da Prefeitura de São Paulo que organiza e desenvolve diversos projetos com o objetivo de melhorar a educação na cidade de São Paulo~\cite{Patio}.

As iniciativas do Pátio Digital se organizam em três eixos:

\begin{itemize}
    \item Transparência e Dados Abertos
    \item Colaboração Governo-Sociedade
    \item Inovação Tecnológica
\end{itemize}

Abrangendo estes três eixos, algumas das iniciativas do Pátio Digital ocorrem no formato de projetos de software de código aberto desenvolvidos por meio do Github, uma plataforma de desenvolvimento colaborativo~\cite{Github}.

%------------------------------------------------

\section*{Atividade e Objetivos}
A minha participação na disciplina ocorreu na forma de colaborações a alguns projetos de código aberto do Pátio Digital disponibilizados no Github.

Minhas colaborações ocorreram na forma de código, isto é, enviando Pull Requests~\cite{PR}, na forma de debates e discussões diretamente com o time e por meio do Github e, também, na forma de revisão de Pull Requests enviados por outros colaboradores no Github.  O tempo investido nestas atividades foi registrado em um blog criado por mim dedicado a documentar e disponibilizar estes dados~\cite{Blog}.

O objetivo do projeto era resolver as issues~\cite{Issue} atribuídas a mim referentes aos projetos nos quais eu contribuí.

%------------------------------------------------

\section*{Fila da Creche}
O Fila da Creche~\cite{Fila} é um aplicativo web voltado para responsáveis de crianças em idade de creche que estão buscando compreender como funciona o processo de cadastro em uma fila de creche na cidade de São Paulo e obter informações sobre a situação das filas de seu interesse.

O site é uma aplicação single-page~\cite{SinglePage}. Neste projeto eu trabalhei apenas com código front-end~\cite{FrontEnd}. As tecnologias envolvidas nas minhas contribuições foram principalmente~\texttt{React.js}~\cite{React} e~\texttt{CSS}~\cite{CSS}.

Iniciei minhas atividades neste projeto com contribuições principalmente no sentido de refatoração~\cite{C0}. Resolvendo a demanda gerada pela Issue 6~\cite{I6}.

Durante estas aplicações, foi realizado um estudo e uma discussão sobre algumas mudanças de interface sugeridas por mim~\cite{I27,I35}. Estas discussões levaram a algumas alterações, implementadas por mim, na interface das páginas de Resultados e Registro~\cite{I34}.

Após algum tempo, foi notado que os \textit{lint warnings}~\cite{Lint} estavam dificultando o processo de desenvolvimento~\cite{I45}. \textit{Lint warnings} são avisos que ocorrem quando o aplicativo é montado que podem ser muito úteis para evitar \textit{bugs}. O aplicativo possuía vários desses avisos que eram ignorados pelos desenvolvedores por não serem graves, o que dificultava a identificação de novos avisos. Eu resolvi as demandas geradas pelos avisos.

Houve também uma demanda, desde o início da minha participação no projeto, de se implementar testes~\cite{I8,I9}. Eu contribuí com várias discussões sobre quais testes deveriam ser implementados e, também, implementando alguns.

Eu implementei testes funcionais para funções auxiliares importantes do código~\cite{P40,P41,P42}, inclusive encontrando um \textit{bug} ao criar testes para uma destas funcionalidades. Além dos testes funcionais, foi implementado um grande teste \textit{End-to-End}~\cite{P43}.

Após estas contribuições, realizei ainda alguns outros ajustes~\cite{P44,P49,P51} e revisei pull requests de outros contribuidores~\cite{P52,P56,P57,P53}.

%------------------------------------------------

\section*{Plataforma Currículo}

O Plataforma Currículo é uma outra iniciativa da SME. Este também é um site e visa ser uma versão digital e interativa do Currículo da Cidade de São Paulo~\cite{Curriculo}. Ela tem funcionalidades bem mais robustas do que o Fila da Creche e disponibiliza, aos professores e alunos, informações e instruções sobre atividades didáticas a serem realizadas como parte do ensino.  

Cada atividade possui uma documentação organizada sobre seus metadados, como qual é o tempo estimado da atividade (em aulas), quais são os componentes relacionados à atividade (português, matemática, etc), informações bibliográficas e ainda outras informações relacionadas a conceitos mais avançados criados pelo próprio currículo como Objetivos de Desenolvimento Sustentável~\cite{ODS}, Matriz de Saberes~\cite{Matriz} e Objetivos de Aprendizagem~\cite{Objetivos}.

Minhas contribuições tiveram um aspecto um pouco mais amplo do que as no outro projeto. Eu contribui não só no frond-end realizando melhorias e ajustes nas interfaces, mas também no back-end, adicionando funcionalidade nova à plataforma.

O código-fonte deste projeto está distribuido, no GitHub, em três repositórios: O \texttt{SME-plataforma-curriculo-interface}~\cite{PCInterface}, que contém a implementação do front-end que é exposto ao usuário final em \texttt{React.js}, o \texttt{SME-plataforma-curriculo-api}~\cite{PCAPI}, que contém a implementação do administrador do conteúdo e é baseado na framework Active Admin~\cite{ActiveAdmin}, de Ruby~\cite{Ruby}, e o \texttt{SME-plataforma-curriculo}~\cite{PC}, que centraliza os dois repositórios, suas issues e alguns scripts que facilitam a execução dos dois concomitantemente. Eu contribui nos três repositórios.

Minhas contribuições se iniciaram, neste projeto, com um ajuste de interface~\cite{PCI_P69} que só envolvia o código de um dos repositórios. Após isso, passei a tentar instalar os outros repositórios, o que foi consideravelmente mais complicado, já que envolvia o uso de docker~\cite{Docker}, tecnologia que eu desconhecia. As instruções do README do projeto haviam me direcionado num caminho errado enquanto eu fazia a configuração do projeto, o que me levou a criar um pull request para melhorá-lo~\cite{PC_P26}. Tal contribuição foi aceita.

Após aprender a configurar o ambiente e documentar este aprendizado, eu estava pronto para começar a resolver os problemas levantados no back-end da plataforma. Eu comecei aceitando a inserção de conteúdo em vídeo por meio do Active Admin~\cite{PC_I20}. Esta contribuição envolveu um estudo tanto sobre o Active Admin quanto sobre o Quill~\cite{Quill}, plugin usado para a formatação de texto rico.

Após permitir a adição de vídeos no administrador, foi necessário fazer com que a interface conseguisse interpretar este tipo de informação, caso contrário, os vídeos adicionados seriam simplesmente ignorados.

Neste momento, foi necessário investigar mais as issues disponíveis. Eu participei de algumas discussões~\cite{PC_I15}, arrumei bugs~\cite{PCA_P52}, entre outros. Até que eu comecei a trabalhar em resolver um bug específico~\cite{PC_I4}. A investigação sobre este bug foi complexa. Existia uma falha difícil de perceber na configuração do Quill que fazia com que conteúdos copiados para o editor rico mantivessem a sua formatação original quando salvos. 

Além destas contribuições, eu revisei uma série de pull requests submetidos aos repositórios, buscando agilizar o processo de evolução do código e incentivar a colaboração da comunidade ao projeto. 

%------------------------------------------------

\section*{Resultados}

O trabalho realizado junto com o Pátio Digital foi extremamente gratificante por uma série de motivos. Primeiramente, e mais importante, estou convencido de que as minhas contribuições foram de grande ajuda para a evolução do projeto, já que os mantenedores se mostraram muito satisfeitos com a minha produtividade e a qualidade do trabalho. Eu me esforcei para suprir as demandas mais importantes para eles de acordo com seus pedidos e também proativamente sugeri outras melhorias. Acredito ter causado um impacto relevante do qual me orgulho.

Além de me sentir grato por ter colaborado, sinto que aprendi muito sobre como a colaboração em um projeto open-source com pouco contato pessoal funciona. Neste sentido, a experiência certamente vai ser muito boa para os eventuais projetos open-source nos quais eu possa vir a colaborar.

Também tive a oportunidade de praticar minha habilidade de revisar código, o que envolve várias habilidades importantes: Ler e compreender código de terceiros, criar e redatar sugestões construtivas para desenvolvedores desconhecidos de maneira encorajadora e educada e pensar nos possíveis problemas e bugs que podem haver ocorrido no desenolvimento de um código até então desconhecido. Competência ao revisar código é extremamente importante tanto para o desenvolvimento open-source quanto para o mercado de trabalho.

Acredito que eu tenha passado uma impressão satisfatória aos membros do Pátio Digital responsáveis pela parceria com a USP, demonstrando que eles podem se alavancar da capacidade de alunos interessados em colaborar com seus projetos. Espero que esta parceria se estenda e que mais alunos tenham a oportunidade de trabalhar com um time que, da minha perspectiva, demonstrou ser dedicado a causar impactos importantes e significativos para o bem da população da cidade de São Paulo, contribuindo para a transparência e acesso à informação.

%------------------------------------------------

\section*{Logs}

Durante o semestre, eu registrei as atividades realizadas na matéria em um Blog disponível em~\texttt{https://victorsenam.github.io/blog-mac0213/}. Segue uma transcrição do contúdo deste blog:

\hypertarget{atuxe9-2-de-setembro}{%
\subsection{Até 2 de Setembro}\label{atuxe9-2-de-setembro}}

\begin{itemize}
\tightlist
\item
  \textbf{0h30} Instalação do ambiente
\item
  \textbf{2h00} Estudo inicial do código, leitura das issues, estudo
  inicial da tecnologia usada.
\item
  \textbf{1h00} Reunião no Pátio Digital com os envolvidos no projeto
\item
  \textbf{0h30}
  \href{https://github.com/prefeiturasp/SME-FilaDaCreche/pull/13}{Alert.js
  as a stateless component}
\item
  \textbf{0h30}
  \href{https://github.com/prefeiturasp/SME-FilaDaCreche/pull/14}{BackButton
  as a stateless component}
\item
  \textbf{0h15}
  \href{https://github.com/prefeiturasp/SME-FilaDaCreche/pull/15}{Banner
  as a stateless component}
\item
  \textbf{0h45}
  \href{https://github.com/prefeiturasp/SME-FilaDaCreche/pull/16}{Creates
  SchoolContact component}
\item
  \textbf{2h00} Criação deste log
\item
  \textbf{0h05}
  \href{https://github.com/prefeiturasp/SME-FilaDaCreche/pull/24}{Removes
  CollapseButton}
\item
  \textbf{0h30}
  \href{https://github.com/prefeiturasp/SME-FilaDaCreche/pull/25}{ContinueButton
  as a stateless component}
\item
  \textbf{0h20}
  \href{https://github.com/prefeiturasp/SME-FilaDaCreche/pull/26}{DefaultButton
  as a stateless component}
\item
  \textbf{0h40}
  \href{https://github.com/prefeiturasp/SME-FilaDaCreche/issues/27}{Estudo
  das possíveis mudanças na interface}
\item
  \textbf{0h15} Atualização do log
\end{itemize}

\hypertarget{semana-de-3-de-setembro}{%
\subsection{Semana de 3 de setembro}\label{semana-de-3-de-setembro}}

\hypertarget{fila-da-creche}{%
\subsubsection{Fila da Creche}\label{fila-da-creche}}

\begin{itemize}
\tightlist
\item
  \textbf{0h19}
  \href{https://github.com/prefeiturasp/SME-FilaDaCreche/pull/28}{Footer
  as a stateless component}
\item
  \textbf{0h09}
  \href{https://github.com/prefeiturasp/SME-FilaDaCreche/pull/29}{HomeBanner
  as a stateless component}
\item
  \textbf{0h02}
  \href{https://github.com/prefeiturasp/SME-FilaDaCreche/pull/30}{Logo
  as a stateless component}
\item
  \textbf{0h04}
  \href{https://github.com/prefeiturasp/SME-FilaDaCreche/pull/31}{SubBanner
  as a stateless component}
\item
  \textbf{0h19}
  \href{https://github.com/prefeiturasp/SME-FilaDaCreche/pull/32}{Spacer
  as a stateless component}
\item
  \textbf{0h11}
  \href{https://github.com/prefeiturasp/SME-FilaDaCreche/pull/33}{ParagraphsList
  as a stateless component}
\item
  \textbf{0h06}
  \href{https://github.com/prefeiturasp/SME-FilaDaCreche/pull/33}{ParagraphsList
  as a stateless component}
\item
  \textbf{0h36} Arrumando este site
\item
  \textbf{1h14} Estudando Continuous Integration
\item
  \textbf{0h52}
  \href{https://github.com/prefeiturasp/SME-FilaDaCreche/issues/27}{Estudando
  possíveis mudanças na interface}
\item
  \textbf{1h41}
  \href{https://github.com/prefeiturasp/SME-FilaDaCreche/issues/8\#issuecomment-419673543}{Estudando
  testes}
\item
  \textbf{0h45}
  \href{https://github.com/prefeiturasp/SME-FilaDaCreche/issues/34}{Implementando
  mudanças na interface}
\item
  \textbf{1h53}
  \href{https://github.com/prefeiturasp/SME-FilaDaCreche/pull/40}{Testes
  e bugfix em utils/toTitleCase}
\item
  \textbf{0h06}
  \href{https://github.com/prefeiturasp/SME-FilaDaCreche/pull/41}{Testes
  em utils/composeDateOfBirthMsg}
\item
  \textbf{0h50}
  \href{https://github.com/prefeiturasp/SME-FilaDaCreche/pull/42}{Testes
  em utils/calculatePreschoolGroup}
\item
  \textbf{1h41} Estudando padrões de React
\end{itemize}

\hypertarget{semana-de-10-de-setembro}{%
\subsection{Semana de 10 de setembro}\label{semana-de-10-de-setembro}}

\begin{itemize}
\tightlist
\item
  \textbf{0h42} Arrumando este site
\item
  \textbf{4h09}
  \href{https://github.com/prefeiturasp/SME-FilaDaCreche/pull/43}{Criando
  teste E2E}
\item
  \textbf{1h37} Estudando testes
\item
  \textbf{1h27}
  \href{https://github.com/prefeiturasp/SME-FilaDaCreche/pull/44}{Issue
  21}
\item
  \textbf{0h49}
  \href{https://github.com/prefeiturasp/SME-FilaDaCreche/pull/46}{Limpando
  variáveis não usadas}
\end{itemize}

\hypertarget{semana-de-17-de-setembro}{%
\subsection{Semana de 17 de setembro}\label{semana-de-17-de-setembro}}

\begin{itemize}
\tightlist
\item
  \textbf{0h15} Atualizando log
\item
  \textbf{0h20} Estudando viabilidade de tipagem
\item
  \textbf{0h58} Investigando
  \href{https://github.com/prefeiturasp/SME-filadacreche/issues/7}{issue
  7}
\item
  \textbf{1h12}
  \href{https://github.com/prefeiturasp/SME-filadacreche/issues/21}{Issue
  21}
\item
  \textbf{0h12}
  \href{https://github.com/prefeiturasp/SME-filadacreche/issues/45}{Issue
  45}
\item
  \textbf{1h05} Me atualizando sobre as discussões no Github
\item
  \textbf{4h15}
  \href{https://github.com/prefeiturasp/SME-FilaDaCreche/pull/48}{Refatorando
  Results.js}
\item
  \textbf{3h15}
  \href{https://github.com/prefeiturasp/SME-FilaDaCreche/pull/51}{Refatorando
  SchoolList.js}
\item
  \textbf{3h02} Estudando padrões de React
\end{itemize}

\hypertarget{semana-de-24-de-setembro}{%
\subsection{Semana de 24 de setembro}\label{semana-de-24-de-setembro}}

\hypertarget{fila-da-creche-1}{%
\subsubsection{Fila da Creche}\label{fila-da-creche-1}}

\begin{itemize}
\tightlist
\item
  \textbf{0h45} Me atualizando sobre as discussões no GitHub
\item
  \textbf{0h35} Estudando padrões de React
\item
  \textbf{2h34}
  \href{https://github.com/prefeiturasp/SME-FilaDaCreche/pull/51}{Refatorando
  SchoolList.js}
\end{itemize}

\hypertarget{plataforma-curruxedculo}{%
\subsubsection{Plataforma Currículo}\label{plataforma-curruxedculo}}

\begin{itemize}
\tightlist
\item
  \textbf{1h33} Estudando projeto
\item
  \textbf{0h27}
  \href{https://github.com/prefeiturasp/SME-plataforma-curriculo-interface}{Instalando
  plataforma-curriculo-interface}
\item
  \textbf{2h52} Estudando issues disponíveis
\item
  \textbf{1h05}
  \href{https://github.com/prefeiturasp/SME-plataforma-curriculo/issues/19}{Issue
  19}
\item
  \textbf{0h25} Estudando \href{http://docker.com/}{docker}
\end{itemize}

\hypertarget{semana-de-1-de-outubro}{%
\subsection{Semana de 1 de outubro}\label{semana-de-1-de-outubro}}

\hypertarget{fila-da-creche-2}{%
\subsubsection{Fila da Creche}\label{fila-da-creche-2}}

\begin{itemize}
\tightlist
\item
  \textbf{0h27}
  \href{https://github.com/prefeiturasp/SME-plataforma-curriculo-interface/pull/70}{Revisando
  Pull Request 70 na interface}
\end{itemize}

\hypertarget{plataforma-curruxedculo-1}{%
\subsubsection{Plataforma Currículo}\label{plataforma-curruxedculo-1}}

\begin{itemize}
\tightlist
\item
  \textbf{2h12} Estudando \href{https://www.docker.com/}{docker}
\item
  \textbf{3h12} Configurando \href{https://www.docker.com/}{docker}
\item
  \textbf{2h03} Estudando issues disponívels
\item
  \textbf{2h23}
  \href{https://github.com/prefeiturasp/SME-plataforma-curriculo/issues/20}{Issue
  20}
\end{itemize}

\hypertarget{semana-de-8-de-outubro}{%
\subsection{Semana de 8 de outubro}\label{semana-de-8-de-outubro}}

\hypertarget{plataforma-curruxedculo-2}{%
\subsubsection{Plataforma Currículo}\label{plataforma-curruxedculo-2}}

\begin{itemize}
\tightlist
\item
  \textbf{1h38} Escrevendo
  \href{https://github.com/prefeiturasp/SME-plataforma-curriculo/pull/26}{instruções
  de instalação no README}
\item
  \textbf{1h21} Configurando \href{https://www.docker.com/}{docker}
\item
  \textbf{1h15} Estudando filtros do
  \href{https://activeadmin.info/}{active admin} para facilitar
  melhorias na
  \href{https://github.com/prefeiturasp/SME-plataforma-curriculo/issues/2}{issue
  2}
\item
  \textbf{1h05} Estudando issues disponíveis
\item
  \textbf{0h12}
  \href{https://github.com/prefeiturasp/SME-plataforma-curriculo-API/pull/52}{Arrumando
  typo de CSS no admin}
\item
  \textbf{0h22}
  \href{https://github.com/prefeiturasp/SME-plataforma-curriculo/issues/15}{Investigando
  issue 15}
\item
  \textbf{0h24}
  \href{https://github.com/prefeiturasp/SME-plataforma-curriculo/issues/16}{Investigando
  issue 16}
\item
  \textbf{0h53} Investigando pergunta da
  \href{https://github.com/prefeiturasp/SME-plataforma-curriculo/issues/4}{issue
  4}. Poruqe alguns links ficam azuis e outros não?
\item
  \textbf{0h43}
  \href{https://github.com/prefeiturasp/SME-plataforma-curriculo/issues/15}{Issue
  15}
\item
  \textbf{0h24}
  \href{https://github.com/prefeiturasp/SME-plataforma-curriculo/issues/2}{Issue
  2}
\item
  \textbf{3h35}
  \href{https://github.com/prefeiturasp/SME-plataforma-curriculo/issues/20}{Issue
  20}
\item
  \textbf{2h25}
  \href{https://github.com/prefeiturasp/SME-plataforma-curriculo/issues/4}{Issue
  4}
\item
  \textbf{0h24} Me atualizando nas issues
\item
  \textbf{0h09} Organizando log
\item
  \textbf{1h17} Estudando padrões de react
\end{itemize}

\hypertarget{semana-de-15-de-outubro}{%
\subsection{Semana de 15 de outubro}\label{semana-de-15-de-outubro}}

\hypertarget{fila-da-creche-3}{%
\subsubsection{Fila da Creche}\label{fila-da-creche-3}}

\begin{itemize}
\tightlist
\item
  \textbf{0h34}
  \href{https://github.com/prefeiturasp/SME-FilaDaCreche/pull/53}{Revisando
  Pull Request 53}
\item
  \textbf{0h48}
  \href{https://github.com/prefeiturasp/SME-FilaDaCreche/pull/57}{Revisando
  Pull Request 57}
\item
  \textbf{0h43}
  \href{https://github.com/prefeiturasp/SME-FilaDaCreche/pull/56}{Revisando
  Pull Request 56}
\item
  \textbf{0h15}
  \href{https://github.com/prefeiturasp/SME-FilaDaCreche/pull/52}{Revisando
  Pull Request 52}
\end{itemize}

\hypertarget{semana-de-22-de-outubro}{%
\subsection{Semana de 22 de outubro}\label{semana-de-22-de-outubro}}

\begin{itemize}
\tightlist
\item
  \textbf{2h33} Atualizando logs
\item
  \textbf{1h21} Pôster
\end{itemize}

\hypertarget{fila-da-creche-4}{%
\subsubsection{Fila da Creche}\label{fila-da-creche-4}}

\begin{itemize}
\tightlist
\item
  \textbf{2h18} Revisando Pull Requests e discussões no GitHub
\item
  \textbf{2h20} Estudando Design Patterns de React
\item
  \textbf{0h31} Investigando Bug da
  \href{https://github.com/prefeiturasp/SME-filadacreche/issues/58}{Issue
  58}
\end{itemize}

\hypertarget{plataforma-curruxedculo-3}{%
\subsubsection{Plataforma Currículo}\label{plataforma-curruxedculo-3}}

\begin{itemize}
\tightlist
\item
  \textbf{0h42} Estudando filtros do
  \href{https://activeadmin.info/}{Active Admin}
\item
  \textbf{0h15} Estudando issues disponíveis
\item
  \textbf{1h14} Revisando Pull Requests e discussões no GitHub
\end{itemize}

\hypertarget{semana-de-29-de-outubro}{%
\subsection{Semana de 29 de outubro}\label{semana-de-29-de-outubro}}

\begin{itemize}
\tightlist
\item
  \textbf{1h43} Pôster---
\end{itemize}

\hypertarget{semana-de-5-de-novembro}{%
\subsection{Semana de 5 de novembro}\label{semana-de-5-de-novembro}}

\begin{itemize}
\tightlist
\item
  \textbf{1h20} Atualizando logs
\item
  \textbf{4h14} Pôster
\end{itemize}

\hypertarget{fila-da-creche-5}{%
\subsubsection{Fila da Creche}\label{fila-da-creche-5}}

\begin{itemize}
\tightlist
\item
  \textbf{0h15} Me atualizando sobre discussões no GitHub
\end{itemize}

\hypertarget{semana-de-26-de-novembro}{%
\subsection{Semana de 26 de novembro}\label{semana-de-26-de-novembro}}

\begin{itemize}
\tightlist
\item
  \textbf{3h16} Relatório
\end{itemize}


\begin{thebibliography}{9}
\bibitem{Patio}
[online] http://patiodigital.prefeitura.sp.gov.br/o-que-e-o-patio-digital/ [Acessado em 28 Nov. 2018]

\bibitem{Github}
[online] https://github.com [Acessado em 28 Nov. 2018]

\bibitem{PR}
[online] https://help.github.com/articles/about-pull-requests/ [Acessado em 28 Nov. 2018]

\bibitem{Blog}
[online] https://victorsenam.github.io/blog-mac0213/ [Acessado em 28 Nov. 2018]

\bibitem{Issue}
[online] https://guides.github.com/features/issues/ [Acessado em 28 Nov. 2018]

\bibitem{Fila}
[online] http://filadacreche.sme.prefeitura.sp.gov.br [Acessado em 28 Nov. 2018]

\bibitem{SinglePage}
[online] http://blog.locaweb.com.br/artigos/desenvolvimento-artigos/o-que-e-single-page-application/ [Acessado em 28 Nov. 2018]

\bibitem{FrontEnd}
[online] https://www.treinaweb.com.br/blog/o-que-e-front-end-e-back-end/ [Acessado em 28 Nov. 2018]

\bibitem{React}
[online] https://reactjs.org [Acessado em 28 Nov. 2018]

\bibitem{CSS}
[online] https://developer.mozilla.org/pt-BR/docs/Web/CSS [Acessado em 28 Nov. 2018]

\bibitem{C0}
[online] https://github.com/prefeiturasp/SME-FilaDaCreche/commits?author=victorsenam&since=2018-08-01&until=2018-09-01 [Acessado em 28 Nov. 2018]

\bibitem{I6}
[online] https://github.com/prefeiturasp/SME-FilaDaCreche/issues/6 [Acessado em 28 Nov. 2018]

\bibitem{I27}
[online] https://github.com/prefeiturasp/SME-FilaDaCreche/issues/27 [Acessado em 28 Nov. 2018]

\bibitem{I35}
[online] https://github.com/prefeiturasp/SME-FilaDaCreche/issues/35 [Acessado em 28 Nov. 2018]

\bibitem{I34}
[online] https://github.com/prefeiturasp/SME-FilaDaCreche/issues/34 [Acessado em 28 Nov. 2018]

\bibitem{Lint}
[online] https://en.wikipedia.org/wiki/Lint\_(software) [Acessado em 28 Nov. 2018]

\bibitem{I45}
[online] https://github.com/prefeiturasp/SME-FilaDaCreche/issues/45 [Acessado em 28 Nov. 2018]

\bibitem{I8}
[online] https://github.com/prefeiturasp/SME-FilaDaCreche/issues/8 [Acessado em 28 Nov. 2018]

\bibitem{I9}
[online] https://github.com/prefeiturasp/SME-FilaDaCreche/issues/9 [Acessado em 28 Nov. 2018]

\bibitem{P40}
[online] https://github.com/prefeiturasp/SME-FilaDaCreche/pull/40 [Acessado em 28 Nov. 2018]

\bibitem{P41}
[online] https://github.com/prefeiturasp/SME-FilaDaCreche/pull/41 [Acessado em 28 Nov. 2018]

\bibitem{P42}
[online] https://github.com/prefeiturasp/SME-FilaDaCreche/pull/42 [Acessado em 28 Nov. 2018]

\bibitem{P43}
[online] https://github.com/prefeiturasp/SME-FilaDaCreche/pull/43 [Acessado em 28 Nov. 2018]

\bibitem{P44}
[online] https://github.com/prefeiturasp/SME-FilaDaCreche/pull/44 [Acessado em 28 Nov. 2018]

\bibitem{P49}
[online] https://github.com/prefeiturasp/SME-FilaDaCreche/pull/49 [Acessado em 28 Nov. 2018]

\bibitem{P51}
[online] https://github.com/prefeiturasp/SME-FilaDaCreche/pull/51 [Acessado em 28 Nov. 2018]

\bibitem{P52}
[online] https://github.com/prefeiturasp/SME-FilaDaCreche/pull/52 [Acessado em 28 Nov. 2018]

\bibitem{P56}
[online] https://github.com/prefeiturasp/SME-FilaDaCreche/pull/56 [Acessado em 28 Nov. 2018]

\bibitem{P57}
[online] https://github.com/prefeiturasp/SME-FilaDaCreche/pull/57 [Acessado em 28 Nov. 2018]

\bibitem{P53}
[online] https://github.com/prefeiturasp/SME-FilaDaCreche/pull/53 [Acessado em 28 Nov. 2018]

\bibitem{Curriculo}
[online] http://curriculo.prefeitura.sp.gov.br/curriculo [Acessado em 28 Nov. 2018]

\bibitem{ODS}
[online] http://curriculo.prefeitura.sp.gov.br/ods [Acessado em 28 Nov. 2018]

\bibitem{Matriz}
[online] http://curriculo.prefeitura.sp.gov.br/matriz-de-saberes [Acessado em 28 Nov. 2018]

\bibitem{Objetivos}
[online] http://curriculo.prefeitura.sp.gov.br/objetivos-de-aprendizagem [Acessado em 28 Nov. 2018]

\bibitem{PCInterface}
[online] https://github.com/prefeiturasp/SME-plataforma-curriculo-interface [Acessado em 28 Nov. 2018]

\bibitem{PCAPI}
[online] https://github.com/prefeiturasp/SME-plataforma-curriculo-API [Acessado em 28 Nov. 2018]

\bibitem{ActiveAdmin}
[online] https://activeadmin.info/ [Acessado em 28 Nov. 2018]

\bibitem{Ruby}
[online] https://www.ruby-lang.org/pt/ [Acessado em 28 Nov. 2018]

\bibitem{PC}
[online] https://github.com/prefeiturasp/SME-plataforma-curriculo [Acessado em 28 Nov. 2018]

\bibitem{PCI_P69}
[online] https://github.com/prefeiturasp/SME-plataforma-curriculo-interface/pull/69 [Acessado em 28 Nov. 2018]

\bibitem{Docker}
[online] https://www.docker.com/ [Acessado em 28 Nov. 2018]

\bibitem{PC_P26}
[online] https://github.com/prefeiturasp/SME-plataforma-curriculo/pull/26 [Acessado em 28 Nov. 2018]

\bibitem{PC_I20}
[online] https://github.com/prefeiturasp/SME-plataforma-curriculo/issues/20 [Acessado em 28 Nov. 2018]

\bibitem{Quill}
[online] https://github.com/quilljs/quill [Acessado em 28 Nov. 2018]

\bibitem{PC_I15}
[online] https://github.com/prefeiturasp/SME-plataforma-curriculo/issues/15 [Acessado em 28 Nov. 2018]

\bibitem{PCA_P52}
[online] https://github.com/prefeiturasp/SME-plataforma-curriculo-api/pull/52 [Acessado em 28 Nov. 2018]

\bibitem{PC_I4}
[online] https://github.com/prefeiturasp/SME-plataforma-curriculo/issues/4 [Acessado em 28 Nov. 2018]

\end{thebibliography}

\end{document}
